\section{Single Population}
In this chapter we consider (the univariate case) a population of individuals $N(t)$ which is a continuous differentiable function of time.\\
We might worry that $N$ should be integer values, i.e. not continuous.
\subsection{Linear Growth Model (Mattheus 1798)}
We suppose that our birth rate is proportional to our population size and is $bN(t)$ with $b > 0$\\
and also suppose our death rate is also proportional and is $dN(t)$ with $d > 0$\\
thus
\begin{align*}
  \frac{dN(t)}{dt} &= bN(t) - dN(t) = (b-d)N(t) = rN(t) = rN \\
  N(t) &= \int rN(t)\\
  N(t) &= N_{0}e^{rt}
\end{align*}
\textit{Note: N(t) will often be written as N from here on}\\

There are issues with physical plausability of this model if $r > 0$ as diverging populations are rarely observed over a sustained period. Likewise $r < 0$ then we can have $n < 1$ which also doesn't really make any sense.

\subsection{Logistic Growth Model (Verhulst 1836)}
The intuition here is that excess food $\implies$ population growth. \\
\\
It seems reasonable to modify our growth model to have the birth rate depend on population size. Let's suppose that $F_{0}$ is a food provision rate and food is consumed at a rate proportional to population size: $eN$.
Thus the rate of food excess is $F_{0} - eN$ and we suppose that per capita birth rate is proportional to this. \\
Such that
\begin{align*}
  \frac{dN}{dt} &= \beta(F_{0} - eN)N && \beta\text{, }e > 0 \\
  \frac{dN}{dt} &= \beta F_{0}(1 - \frac{eN}{F_{0}})N = r^{*}(1 - \frac{N}{k})N &&r^{*} = \beta F_{0}\text{, }\frac{1}{k} = \frac{e}{F_{0}}
\end{align*}
\textit{Note: when $\frac{N}{k} \to 0$ we see that $\frac{dN}{dt} \to r^{*}N$ so we get our linear growth model back}\\

$r^{*}$ can be thought of as the linear replication rate in the setting that N is small (compared to k) and k can be thought of as a carrying capacity or typical population size.

\subsubsection*{Exact Analytic Solution}
TODO...
