\usepackage{fullpage} % Package to use full page
\usepackage{parskip} % Package to tweak paragraph skipping
\usepackage{tikz} % Package for drawing
\usetikzlibrary{shapes,positioning,fit,calc}
\usepackage{mathtools}
\usepackage{amssymb}
\usepackage[hidelinks]{hyperref}
\usepackage{xifthen}
\usepackage{dsfont}
\usepackage{xparse}
\usepackage{etoc}
\usepackage{tcolorbox}
\usepackage{enumitem}
\usepackage{float}
\usepackage{bookmark}
\DeclareDocumentCommand{\set}{m o o}{%
\IfNoValueTF{#2}{%
%no second argument
\mathds{#1}
}{%
\IfNoValueTF{#3}{%
%no third argument
\mathds{#1}^{#2}
}{%
%all argument exists
\mathds{#1}^{#2\times #3}
}%
}%
}
\DeclareDocumentCommand{\inset}{m o o}{%
\in \set{#1}[#2][#3]
}
\DeclareDocumentCommand{\addlabel}{m m}{\underbrace{#1}_{\mathclap{\text{#2}}}}
\DeclareDocumentCommand{\addtoplabel}{m m}{\overbrace{#1}^{\mathclap{\text{#2}}}}
\DeclareDocumentCommand{\fixedlabel}{m m O{70pt}}{\underbrace{#1}_{\parbox{#3}{\centering#2}}}
\DeclareDocumentCommand{\fixedtoplabel}{m m O{70pt}}{\overbrace{#1}^{\parbox{#3}{\centering#2}}}
\DeclareDocumentCommand{\stretchlabel}{m m}{\underbrace{#1}_{\text{#2}}}
\DeclareDocumentCommand{\stretchtoplabel}{m m}{\overbrace{#1}^{\text{#2}}}
\DeclareDocumentCommand{\custombox}{m m m o O{black!50!gray} O{white} O{black!50!gray}}{%
\IfNoValueTF{#4}{%
\begin{tcolorbox}[title={\subsubsection{#1: #2}}, colbacktitle=#5, coltitle=#6, colframe=#7]%
  #3%
\end{tcolorbox}%
}{%
\begin{tcolorbox}[title={\subsubsection{#1: #2}}, label=#4, colbacktitle=#5, coltitle=#6, colframe=#7]%
  #3%
\end{tcolorbox}%
}%
}
\DeclareDocumentCommand{\example}{m m o}{\custombox{Example}{#1}{#2}[#3]}
\DeclareDocumentCommand{\definition}{m m o}{\custombox{Definition}{#1}{#2}[#3][black!50!red][white][black!50!red]}
\DeclareDocumentCommand{\theorem}{m m o}{\custombox{Theorem}{#1}{#2}[#3][black!25!gray][white][black!25!gray]}
\newcommand\aug{\fboxsep=-\fboxrule\!\!\!\fbox{\strut}\!\!\!}
\newcommand\haug{\rule[.5ex]{3.5em}{0.4pt}}
\DeclareDocumentCommand{\matrix}{m}{\begin{bmatrix}#1\end{bmatrix}}
\DeclareDocumentCommand{\note}{m O{gray}}{\textcolor{#2}{#1}}
\DeclareDocumentCommand{\noteit}{m O{gray}}{\textit{\note{#1}{#2}}}
\DeclareDocumentCommand{\where}{m}{&& \text{Where #1}}
\DeclareDocumentCommand{\norm}{m O{}}{\lVert#1\rVert_{#2}}
\DeclareDocumentCommand{\abs}{m}{\left|#1\right|}
\DeclareDocumentCommand{\vectorset}{m O{n}}{\{#1_{1}, \dots, #1_{#2}\}}
\DeclareDocumentCommand{\vectorsetextra}{m O{n}}{\{#1_{1}, #1_{2}, \dots, #1_{#2}\}}
\DeclareDocumentCommand{\centre}{m}{\begin{center}#1\end{center}}
